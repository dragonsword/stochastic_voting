% !TEX root = stochastic_voting.tex
%%%%%%%%%%%%%%%%%%%%%%%%%%%%%

\section{Model Description}
\label{section:model}

\noindent
In this model, we assume that there are $N = \{1,\dots,N\}$ voters and $C=\{c_1,\dots, c_M\}$ candidates. 
Each voter $i$ submits a vote $\mathbf{v}_i = (r_i(c_1), r_i(c_2), \dots, r_i(c_M))$, where $r_i$ is a scoring rule that assigns a (voting) score to each candidate $c_m$. 
Generally speaking, $r_i(c_m)$ represents how much voter $i$ favours candidate $c_m$.
That is, if the voter prefers candidate $c_{m_1}$ against $c_{m_2}$, then $r_i(c_{m_1}) > r_i(c_{m_2})$.

\textbf{TODO: more precise definition of scoring rule}

In our model, we assume that we do not know the exact realisation of the votes $\mathbf{v}_1,\dots, \mathbf{v}_N $ (e.g., the election has not started yet).
However, we know that each vote $\mathbf{v}_i$ is drawn from a \emph{known} distribution $D(r)$ (e.g., we have the results of the exit polls).
Suppose that we control a set of $S = \{s_1,\dots,s_K\}$ bribed voters (apart from the abovementioned $N$ voters), whose votes can be manipulated in any particular way.
In the present model, we do not focus on the incentives and the techniques of bribery.
Instead, we assume that these bribed voters are already motivated to follow our orders.
\textbf{TODO: need some citations to bribery techniques and incentives.} 
\textbf{We also need to define the vote manipulation algorithm over $S$.} 

We assume that this is a single winner election system, with $c_1$ to be our favoured candidate.
That is, we aim to manipulate the votes of the bribed voters $S$ such that the probability that $c_1$ wins the elections is sufficiently high.
In particular, we focus on the following two problems:

\begin{enumerate}
\item Given $N$, $C$, scoring rule $r$, set of bribed voters $S$ and a manipulation algorithm, what is the probability that $c_1$ can win the elections.
\item Given $N$, $C$, scoring rule $r$, $0 < \alpha < 1$, and a manipulation algorithm, what is the minimum number of voters we have to bribe in order to guarantee that $c_1$ can win the elections with at least $\alpha$ probability.
\end{enumerate}

\textbf{TODO list:}
\begin{itemize}
\item talk about the scoring rules we use here: stick with the quasi-Borda one...
\item manipulation algorithms: reverse, uniform, + something else...
\item say that due to lack of space, we ignore the other scoring rules + algorithms
\end{itemize}
